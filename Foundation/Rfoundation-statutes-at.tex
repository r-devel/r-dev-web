\documentclass[a4paper]{article}

\usepackage[latin1]{inputenc}
\usepackage[austrian]{babel}

\title{\bf Statuten des Vereins\\
  "`The R Foundation for Statistical Computing"'}

\newcommand{\RF}{"`R Foundation"'}

\begin{document}

\maketitle

\section{Name, Sitz und T{\"a}tigkeitsbereich}

\begin{enumerate}
  
 \item Der Verein f{\"u}hrt den Namen "`The R Foundation for Statistical
  Computing"', abgek{\"u}rzt \RF{}, im Folgenden wird die Kurzform verwendet.
 \item Er hat seinen Sitz in Wien und erstreckt seine T{\"a}tigkeit auf
  die gesamte Welt.
\end{enumerate}

\section{Zweck}

\begin{enumerate}
 \item Grunds{\"a}tzliches
  \begin{enumerate}
   \item Die Aktivit{\"a}ten der \RF{} und ihrer Mitglieder sind
    unbeinflu{\ss}t von Politik und Religion, Personen k{\"o}nnen unabh{\"a}ngig
    von Staatsb{\"u}rgerschaft, Konfession oder Wohnsitz Mitglieder
    werden.
    
   \item Die \RF{} ist ein gemeinn{\"u}tziger Verein, dessen T{\"a}tigkeit
    nicht auf Gewinn gerichtet ist.
  \end{enumerate}
 \item Die Ziele der \RF{} sind
  \begin{enumerate}
   \item F{\"o}rderung des "`R Project for Statistical Computing"' um eine
    freie Open Source Softwareumgebung f{\"u}r Datenanalyse und Graphik zur
    Verf{\"u}gung zu stellen.
    
   \item Offizielle Stimme des "`R Project"' zur Kommunikation mit
    Presse sowie an R interessierten gewerblichen und
    nichtgewerblichen Organisationen.

   \item Besitz und Verwaltung des Copyrights der Software R sowie der
    zugeh{\"o}rigen Dokumentation.
  \end{enumerate}
\end{enumerate}

\section{Mittel zur Erreichung der Vereinszwecks}

\begin{enumerate}
 \item Zur Erreichung des Vereinszwecks dienen insbesonders
  \begin{enumerate}
   \item Unterst{\"u}tzung der weiteren Entwicklung von R und {\"a}hnlicher
    Open Source Software Projekte. 
   \item Koordination von Forschungsprojekten und Unterst{\"u}tzung der
    Kommunikation zwischen Anwendern von R, Organisation und
    F{\"o}rderung von Kursen.
   \item Betrieb von Internet-Informationssystemen wie Email, FTP oder
    HTTP Servern.
   \item Organisation und F{\"o}rderung von R-bezogenen Konferenzen und
    Workshops, Repr{\"a}sentation von R auf relevanten anderen Konferenzen,
    sowie {\"O}ffentlichkeitsarbeit f{\"u}r die Anwendung und Entwicklung von
    R und R-bezogener Software.
   \item Publikation von Handb{\"u}chern, technischen Standards,
    Zeitschriften, Artikel und anderen R-bezogenen Dokumenten in
    gedruckter und elektronischer Form.
   \item Vergabe und Verwaltung von Lizenzen f{\"u}r R und zugeh{\"o}rige
    Dokumentation. 
  \end{enumerate}

 \item Die erforderlichen materiellen Mittle sollen aufgebracht werden
  durch Mitgliedsbeitr{\"a}ge, Spenden, und anderen Zuwendungen. M{\"o}gliche
  weitere Einnahmequellen sind Registrations- und Lizenzgeb{\"u}hren.
\end{enumerate}

\section{Mitgliedschaft}

\begin{enumerate}

 \item Die \RF{} besteht aus ordentlichen und unterst{\"u}tzenden
  Mitgliedern. Ordentliche Mitglieder haben eine Stimme in der
  Hauptversammlung und beteiligen sich aktiv an der Arbeit des
  Vereins. Unterst{\"u}tzende Mitglieder haben keine Stimme und f{\"o}rdern den
  Verein prim{\"a}r durch die Zahlung von Mitgliedsbeitr{\"a}gen.

 \item Nur physischen Personen k{\"o}nnen ordentliche Mitglieder
  werden. Neue ordentliche Mitglieder k{\"o}nnen nur durch einen
  Mehrheitsbeschlu{\ss} der bestehenden ordentlichen Mitglieder
  aufgenommen werden. Die Abstimmung {\"u}ber die Zulassung eines neuen
  ordentlichen Mitglieds kann entweder bei einer Hauptversammlung der
  \RF{} oder mit Hilfe elektronischer Kommunikationsmittel
  durchgef{\"u}hrt werden. Die urspr{\"u}nglichen ordentlichen Mitglieder bei
  Gr{\"u}ndung des Vereins bestehen aus den Mitgliedern des "`R
  Development Core Team"', die im Quellcode von R Version 1.5.0
  genannt sind.
  
 \item Jede physische oder juristische Person kann unterst{\"u}tzendes
  Mitglied werden. Neue unterst{\"u}tzende Mitglieder k{\"o}nnen vorl{\"a}ufig vom
  Vereinsvorstand aufgenommen werden. Diese vorl{\"a}ufige Aufnahme mu{\ss}
  durch die Hauptversammlung best{\"a}tigt werden. Die Aufnahme oder
  Best{\"a}tigung der Mitgliedschaft kann ohne {\"o}ffentliche Begr{\"u}ndung
  verweigert werden.
  
 \item Die Mitgliedschaft endet durch
  \begin{enumerate}
   \item Tod, bei juristischen Personen durch Verlust der
   Rechtspers{\"o}nlichkeit. 
   \item freiwilligen Austritt durch schriftliche Mitteilung an den
    Vereinsvorstand.
   \item durch Ausschlu{\ss} aufgrund eines mit Zweidrittelmehrheit
    gefa{\ss}ten Beschlusses der ordentlichen Mitglieder.
  \end{enumerate}
\end{enumerate}
   

\section{Vereinsorgane}

Die Organe der \RF{} sind:
\begin{enumerate}
 \item die Hauptversammlung,
 \item der Vorstand,
 \item die Rechnungspr{\"u}fer,
 \item sowie das Schiedsgericht.
\end{enumerate}


\section{Die Hauptversammlung}
\label{GA}


\begin{enumerate}
 \item Ein wie in Absatz~\ref{GA}.\ref{GAmeet} beschriebenes Treffen
  der ordentlichen Mitglieder der \RF{} oder deren Stimmabgabe wie in
  Ansatz~\ref{GA}.\ref{GAvote} beschrieben stellen die
  Hauptversammlung und somit h{\"o}chste Authorit{\"a}t der \RF{} dar. Jedes
  ordentliche Mitglied hat eine Stimme in der Hauptversammlung.
  
 \item \label{GAmeet}
  Ein Treffen der Hauptversammlung mu{\ss} mindestens einmal alle zwei
  Jahre stattfinden. Die Einladung mit Angabe von Ort, Datum, Uhrzeit
  und Tagesordnung des Treffens der Hauptversammlung hat an alle
  ordentlichen Mitglieder zumindest 
  
  A meeting of the general assembly has to take place at least
  once every two years. A call, specifying the place, date, time and
  the agenda of a general assembly shall be sent to all ordinary
  members not less than one month before the date of the meeting of
  the general assembly. The meeting constitutes a quorum if at least
  two thirds of all ordinary members are present or have sent an
  authorized representative.
  
 \item A meeting of the general assembly has to be called upon request
  by two members of the board or one quarter of all ordinary members. 
  
  \item \label{GAvote}
   In addition to meetings of the general assembly, decisions can
  be reached by casting votes using mail, fax, or e-mail. Questions
  under reference shall be sent to all ordinary members not less than
  one month before the date by which the replies have to reach the
  board of the organization.
  
 \item All decisions of the general assembly are reached by majority
  vote, unless otherwise stated in these statutes.
  
 \item The business transactions of the general assembly include:
  \begin{enumerate}
   \item Election and dismissal of the members of the board.
   \item Election and dismissal of the auditors.
   \item Acceptance of activity report, statement and estimates of
    account.
   \item Release of the board
   \item Determination of membership fees.
   \item Approval or rejection of proposed changes to these statutes.
   \item The decision to terminate the \RF{}.
   \item All other business necessary for the promotion of the
    objectives of the \RF{}.
  \end{enumerate}
\end{enumerate}

\section{Board}

\begin{enumerate}
  \item
   The board of the organization consists of at least four
   persons:
   \begin{enumerate}
    \item Either a president and a vice-president or two presidents of
     equal rights.
    \item A secretary general.
    \item A treasurer.
   \end{enumerate}
   Optionally a vice-secretary and a vice-treasurer may be elected if
   necessary.
   
  \item All members of the board are elected by the general assembly
   for a term of office of two years, reelection is possible.
   
  \item Decisions of the board are by majority vote.
   
  \item The board manages the business of the \RF{} and deals with all
   tasks not assigned to other organs by the statutes, especially
   \begin{enumerate}
    \item Preparation of activity report, statement and estimates of
     account.
    \item Preparation of and call for general assemblies.
    \item Management of all assets.
  \end{enumerate}
  
 \item The president represents the \RF{} on official occasions,
  presides over meetings and is responsible for the overall direction
  of the activities of the \RF{}.
  
 \item The secretary general coordinates the activities of the
  organization, supports the president in leading the organization,
  and keeps records of all general assemblies and decisions of the
  board.
  
 \item The treasurer is responsible for the accounting and asset
  management of the organization.
  
 \item Written contracts of the \RF{} have to be signed by the
  president and the secretary general, contracts concerning financial
  transactions have to be signed by the president and the
  treasurer. Business transactions between members of the board and the
  \RF{} have to be approved by the general assembly.
\end{enumerate}


\section{Auditors}

\begin{enumerate}
 \item Two auditors are elected by the general assembly for a term of
  office of two years, reelection is possible.
 \item The auditors routinely check business and accounting of the
  organization and report to the general assembly.
\end{enumerate}

\section{Termination of Offices}

The office of a member of the board or auditor terminates by
\begin{enumerate}
 \item death of the person or end of term
 \item dismissal by the general assembly
 \item voluntary withdrawal through written notice to the general
  assembly, which takes effect on election of a successor.
\end{enumerate}

\section{Court of Arbitration}

\begin{enumerate}
 \item Disputes between members of the organization are settled by a
  court of arbitration. Each party of the dispute nominates two
  ordinary members as referees, these four elect a fifth member as
  chairman of the court.  If no agreement on the fifth member can be
  obtained, the decision between all candidates is made at random. The
  court of arbitration decides with majority vote, the chairman
  decides in case of a draw due to abstention.

\end{enumerate}

\section{Termination}

\begin{enumerate}
 \item Only a special meeting of the general assembly may decide to
  terminate the organization by an affirmative vote of a two-thirds
  majority of the ordinary members.  A call for the meeting has to be
  made four weeks in advance to the general assembly, the agenda
  accompanying the call must include the termination as a topic.
  
 \item This general assembly has to decide on a recipient for all
  assets, if any, of the organization after liquidation. The recipient
  should be an organization with similar goals as the \RF{}.

 \item The last board is responsible to inform the Austrian
  authorities of the termination and publish it according to the law.
\end{enumerate}


\end{document}

%%% Local Variables: 
%%% mode: latex
%%% TeX-master: t
%%% ispell-local-dictionary: "deutsch8"
%%% End: 
