\documentclass[a4paper]{article}

\title{\bf Statutes of
  ``The R Foundation for Statistical Computing''}

\newcommand{\RF}{``R Foundation''}

\begin{document}

\maketitle

\section{Name, Seat and Field of Activity}

\begin{enumerate}
  \item The organization is named ``The R Foundation for Statistical
   Computing'', abbreviated as \RF{}, which will be used
   throughout this document.
 \item It is seated in Vienna, Austria, and is active worldwide.
\end{enumerate}

\section{Objectives}

\begin{enumerate}
 \item Fundamentals
  \begin{enumerate}
   \item The activities of the \RF{} and its members are not
    influenced by politics or religion; people may become members
    regardless of nationality, religion, or place of residency.
   \item The \RF{} is a non-profit organization working in
    the public interest.
  \end{enumerate}
 \item The objectives of the \RF{} are:
  \begin{enumerate}
   \item Advance the R project for statistical computing to provide a
    free and open source software environment for data analysis and
    graphics. 
   \item Act as an official voice for the R project, providing means of
    communication with the press, commercial and noncommercial
    organizations interested in R.
   \item Hold and administer the copyright of the R software and
    documentation.
  \end{enumerate}
\end{enumerate}

\section{Means to Meet the Objectives}

\begin{enumerate}
 \item To meet its objectives the organization will especially
  \begin{enumerate}
   \item Support the further development of R and related open source
    software projects.
   \item Coordinate research projects, support communication within
    the R user community, and organize or sponsor courses.
   \item Operate Internet information systems like email, FTP or HTTP
    Servers.
   \item Organize and sponsor R-related conferences and workshops,
    represent R at relevant conferences sponsored by others, and
    promote the use and development of R and R-related software.
   \item Publish manuals, technical standards, periodicals, journal
    articles and other R-related documents in printed and electronic
    form.
    \item Issue and administer licenses for the R software and
     documentation.
  \end{enumerate}
  
 \item The material means of the organization to meet its objectives
  will come from membership fees, donations, and contributions.
  Additional sources of income are registration and license fees, if
  any.
\end{enumerate}

\section{Membership}

\begin{enumerate}
 \item The \RF{} consists of ordinary and supporting
  members.  Ordinary members have a vote in the general assembly and
  participate actively in the work of the organization. Supporting
  members have no vote and promote the organization primarily by
  paying membership fees.
  
 \item Only people can become ordinary members. New ordinary
  members shall be admitted only by a majority vote of the existing
  ordinary members. This vote can be conducted either at a general
  assembly of the \RF{} or by electronic means.  The
  initial set of ordinary members at establishment of the organization
  will consist of the members of the ``R Development Core Team'' as
  listed in the sources of R release 1.5.0.
  
 \item Any person or legal entity may become a supporting member.  New
  supporting members can be temporarily admitted by the board of the
  organization.  This temporary admission must be approved by the
  general assembly.  Admission or approval for membership can be
  rejected without public justification.

 \item Membership terminates
  \begin{enumerate}
   \item at the death of a person or the termination of
    existence of legal entities.
   \item by voluntary withdrawal from membership through written
    notice to the board of the organization.
   \item by an affirmative vote of a two-thirds majority of the
    ordinary members. 
  \end{enumerate}
\end{enumerate}
   

\section{Principal Organs}

The principal organs of the \RF{} are: 
\begin{enumerate}
 \item the general assembly
 \item the board
 \item the auditors
 \item and the court of arbitration.
\end{enumerate}


\section{General Assembly}
\label{GA}


\begin{enumerate}
 \item The ordinary members of the \RF{} at a meeting convened in the
  manner specified in paragraph~\ref{GA}.\ref{GAmeet} or casting their
  votes as described in paragraph~\ref{GA}.\ref{GAvote} constitute the
  general assembly and are the highest authority of the \RF{}. Each
  ordinary member has one vote in the general assembly.
  
  \item \label{GAmeet}
   A meeting of the general assembly has to take place at least
   once every two years. A call, specifying the place, date, time and
   the agenda of a general assembly shall be sent to all ordinary
   members not less than one month before the date of the meeting of
   the general assembly. The meeting constitutes a quorum if at least
   two thirds of all ordinary members are present or have sent an
   authorized representative.
  
 \item A meeting of the general assembly has to be called upon request
  by two members of the board or one quarter of all ordinary members. 
  
  \item \label{GAvote}
   In addition to meetings of the general assembly, decisions can
  be reached by casting votes using mail, fax, or e-mail. Questions
  under reference shall be sent to all ordinary members not less than
  one month before the date by which the replies have to reach the
  board of the organization.
  
 \item All decisions of the general assembly are reached by majority
  vote, unless otherwise stated in these statutes.
  
 \item The business transactions of the general assembly include:
  \begin{enumerate}
   \item Election and dismissal of the members of the board.
   \item Election and dismissal of the auditors.
   \item Acceptance of activity report, statement and estimates of
    account.
   \item Release of the board
   \item Determination of membership fees.
   \item Approval or rejection of proposed changes to these statutes.
   \item The decision to terminate the \RF{}.
   \item All other business necessary for the promotion of the
    objectives of the \RF{}.
  \end{enumerate}
\end{enumerate}

\section{Board}

\begin{enumerate}
  \item
   The board of the organization consists of at least four
   persons:
   \begin{enumerate}
    \item Either a president and a vice-president or two presidents of
     equal rights.
    \item A secretary general.
    \item A treasurer.
   \end{enumerate}
   Optionally a vice-secretary and a vice-treasurer may be elected if
   necessary.
   
  \item All members of the board are elected by the general assembly
   for a term of office of two years, reelection is possible.
   
  \item Decisions of the board are by majority vote.
   
  \item The board manages the business of the \RF{} and deals with all
   tasks not assigned to other organs by the statutes, especially
   \begin{enumerate}
    \item Preparation of activity report, statement and estimates of
     account.
    \item Preparation of and call for general assemblies.
    \item Management of all assets.
  \end{enumerate}
  
 \item The president represents the \RF{} on official occasions,
  presides over meetings and is responsible for the overall direction
  of the activities of the \RF{}.
  
 \item The secretary general coordinates the activities of the
  organization, supports the president in leading the organization,
  and keeps records of all general assemblies and decisions of the
  board.
  
 \item The treasurer is responsible for the accounting and asset
  management of the organization.
  
 \item Written contracts of the \RF{} have to be signed by the
  president and the secretary general, contracts concerning financial
  transactions have to be signed by the president and the
  treasurer. Business transactions between members of the board and the
  \RF{} have to be approved by the general assembly.
\end{enumerate}


\section{Auditors}

\begin{enumerate}
 \item Two auditors are elected by the general assembly for a term of
  office of two years, reelection is possible.
 \item The auditors routinely check business and accounting of the
  organization and report to the general assembly.
\end{enumerate}

\section{Termination of Offices}

The office of a member of the board or auditor terminates by
\begin{enumerate}
 \item death of the person or end of term
 \item dismissal by the general assembly
 \item voluntary withdrawal through written notice to the general
  assembly, which takes effect on election of a successor.
\end{enumerate}

\section{Court of Arbitration}

\begin{enumerate}
 \item Disputes between members of the organization are settled by a
  court of arbitration. Each party of the dispute nominates two
  ordinary members as referees, these four elect a fifth member as
  chairman of the court.  If no agreement on the fifth member can be
  obtained, the decision between all candidates is made at random. The
  court of arbitration decides with majority vote, the chairman
  decides in case of a draw due to abstention.

\end{enumerate}

\section{Termination}

\begin{enumerate}
 \item Only a special meeting of the general assembly may decide to
  terminate the organization by an affirmative vote of a two-thirds
  majority of the ordinary members.  A call for the meeting has to be
  made four weeks in advance to the general assembly, the agenda
  accompanying the call must include the termination as a topic.
  
 \item This general assembly has to decide on a recipient for all
  assets, if any, of the organization after liquidation. The recipient
  should be an organization with similar goals as the \RF{}.

 \item The last board is responsible to inform the Austrian
  authorities of the termination and publish it according to the law.
\end{enumerate}


\end{document}

%%% Local Variables: 
%%% mode: latex
%%% TeX-master: t
%%% End: 
